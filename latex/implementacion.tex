\subsection{Eliminación Gaussiana}

\subsubsection{Descripción del método}

El método de Eliminación Gaussiana consiste en una serie de pasos que permiten resolver un sistema de ecuaciones lineales de, en principio, $n$ ecuaciones y $n$ variables.

Sea A $\in \mathbb{R}^{n \times\ n}$ la matriz tal que el elemento en la fila $i$ y columna $j$ ($a_{i,j}$) representa el coeficiente de la variable $j$ en la ecuación $i$.
Y sea b $\in \mathbb{R}^{n}$ el vector tal que el elemento en la fila $i$ ($b_{i}$) representa el termino independiente en la ecuación $i$.

Podemos dividir el método en 2 partes centrales:
\begin{enumerate}
    \item Llevar la matriz A a una forma \textbf{Triangular Superior}, es decir, una matriz equivalente a A tal que tiene ceros debajo de los elementos de la diagonal. El siguiente pseudocódigo muestra como es el algoritmo para realizar esta tarea:

\begin{lstlisting}
Para j desde 0 hasta n-1 hacer:
    Poner pivote = A[j][j]
    Para i desde j+1 hasta n-1 hacer:
        Poner coeficiente = A[i][j] / pivote
        Poner A[i][j] = 0
        Para k desde j+1 hasta n-1 hacer:
            Poner A[i][k] = A[i][k] - coeficiente * A[j][k]
        Fin para
        b[i] = b[i] - coeficiente * b[j]
    Fin para
Fin para
\end{lstlisting}

		Notese que no validamos que la variable ``pivote'' sea distinta de cero. Esto es así ya que por la forma en la que se modeló el problema el pivote siempre es distinto de cero.

    \item \textbf{Resolver el sistema equivalente} que obtuvimos en el paso anterior. Para esto, vamos a utilizar que la matriz es Triangular Superior. La idea es empezar despejando el valor de la $n$-ésima variable, luego usar este valor para despejar la $(n-1)$-ésima variable, y así sucesivamente hasta la primera variable. En pseudocódigo:

\begin{lstlisting}
Poner X = vector de n elementos
Para i desde n-1 hasta 0 hacer:
    Poner X[i] = b[i]
    Para j desde i+1 hasta n-1 hacer:
        Poner X[i] = X[i] - U[i][j] * X[j]
    Fin para
    Poner X[i] = X[i] / U[i][i]
Fin para
\end{lstlisting}

		Donde $U$ es la matriz que calculamos en el paso 1.

  \end{enumerate}
  

\subsubsection{Utilizando que la matriz es Banda}

Si miramos la matriz con la cual representamos el modelo del problema, podemos ver que alrededor de los elementos de la diagonal hay una ``banda'' de tamaño $2n$.
Es decir, si quisieramos poner elementos debajo del elemento $a_{i,i}$, nos bastaría con modificar las filas desde $i+1$ hasta $i+2n+1$, ya que $\forall\ a_{j,i},\ j> i+2n+1 \implies a_{j,i} = 0$.

Usando esto podemos optimizar significativamente el primer paso de la Elminación Gaussiana, que consiste en hallar la matriz equivalente Triangular Superior. El pseudocódigo es el siguiente:

\begin{lstlisting}
Para j desde 0 hasta n-1 hacer:
    Poner pivote = A[j][j]
    Poner inicioBanda = max(i+1, n)
    Poner finBanda = min(n, inicioBanda + n)
    Para i desde inicioBanda hasta finBanda hacer:
        Si A[i][j] != 0 hacer:
            Poner coeficiente = A[i][j] / pivote
            Poner A[i][j] = 0
            Para k desde j+1 hasta n-1 hacer:
                Poner A[i][k] = A[i][k] - coeficiente * A[j][k]
            Fin para
            b[i] = b[i] - coeficiente * b[j]
        Fin si
    Fin para
Fin para
\end{lstlisting}

\subsection{Factorización LU}

\subsubsection{Descripción del método}

\subsection{Determinación de la Isoterma}

Dada la solucion del sistema de ecuaciones, proponer una forma de estimar en cada ángulo de la discretización la posición de la isoterma 500.

\subsubsection{Promedio simple}

\subsubsection{Búsqueda binaria mediante sistemas de ecuaciones}

\subsubsection{Regresión lineal (Linear fit)}


\subsection{Evaluación del peligro de la estructura}

En función de la aproximación de la isoterma, proponer una forma (o medida) a utilizar para evaluar la peligrosidad de la estructura en función de la distancia a la pared externa del horno.
