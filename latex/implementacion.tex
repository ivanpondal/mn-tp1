\subsection{Eliminación Gaussiana}

\subsubsection{Descripción del método}

El método de Eliminación Gaussiana consiste en una serie de pasos que permiten resolver un sistema de ecuaciones lineales de, en principio, $n$ ecuaciones y $n$ variables.

Sea A $\in \mathbb{R}^{n \times\ n}$ la matriz tal que el elemento en la fila $i$ y columna $j$ ($a_{i,j}$) representa el coeficiente de la variable $j$ en la ecuación $i$.
Y sea b $\in \mathbb{R}^{n}$ el vector tal que el elemento en la fila $i$ ($b_{i}$) representa el termino independiente en la ecuación $i$.

Podemos dividir el método en 2 partes centrales:
\begin{enumerate}
    \item Llevar la matriz A a una forma \textbf{Triangular Superior}, es decir, una matriz equivalente a A tal que tiene ceros debajo de los elementos de la diagonal. El siguiente pseudocódigo muestra como es el algoritmo para realizar esta tarea:

\begin{lstlisting}
Para j desde 0 hasta n-1 hacer:
    Poner pivote = A[j][j]
    Para i desde j+1 hasta n-1 hacer:
        Poner coeficiente = A[i][j] / pivote
        Poner A[i][j] = 0
        Para k desde j+1 hasta n-1 hacer:
            Poner A[i][k] = A[i][k] - coeficiente * A[j][k]
        Fin para
        b[i] = b[i] - coeficiente * b[j]
    Fin para
Fin para
\end{lstlisting}

		Notese que no validamos que la variable ``pivote'' sea distinta de cero. Esto es así ya que por la forma en la que se modeló el problema el pivote siempre es distinto de cero.

    \item \textbf{Resolver el sistema equivalente}. Para esto, vamos a utilizar que la matriz es Triangular Superior. La idea es empezar despejando el valor de la $n$-ésima variable, luego usar este valor para despejar la $(n-1)$-ésima variable, y así sucesivamente hasta la primera variable. En pseudocódigo:

\begin{lstlisting}
Poner X = vector de n elementos
Para i desde n-1 hasta 0 hacer:
    Poner X[i] = b[i]
    Para j desde i+1 hasta n-1 hacer:
        Poner X[i] = X[i] - U[i][j] * X[j]
    Fin para
    Poner X[i] = X[i] / U[i][i]
Fin para
\end{lstlisting}

		Donde $U$ es la matriz que calculamos en el paso 1.

  \end{enumerate}


\subsubsection{Utilizando que la matriz es Banda}

Si miramos la matriz con la cual representamos el modelo del problema, podemos ver que alrededor de los elementos de la diagonal hay una ``banda'' de tamaño $2n$.
Es decir, si quisieramos poner elementos debajo del elemento $a_{i,i}$, nos bastaría con modificar las filas desde $i+1$ hasta $i+2n+1$, ya que $\forall\ a_{j,i},\ j> i+2n+1 \implies a_{j,i} = 0$.

Usando esto podemos optimizar significativamente el primer paso de la Elminación Gaussiana, que consiste en hallar la matriz equivalente Triangular Superior. El pseudocódigo es el siguiente:

\begin{lstlisting}
Para j desde 0 hasta n-1 hacer:
    Poner pivote = A[j][j]
    Poner inicioBanda = max(i+1, n)
    Poner finBanda = min(n, inicioBanda + n)
    Para i desde inicioBanda hasta finBanda hacer:
        Si A[i][j] != 0 hacer:
            Poner coeficiente = A[i][j] / pivote
            Poner A[i][j] = 0
            Para k desde j+1 hasta n-1 hacer:
                Poner A[i][k] = A[i][k] - coeficiente * A[j][k]
            Fin para
            b[i] = b[i] - coeficiente * b[j]
        Fin si
    Fin para
Fin para
\end{lstlisting}

\subsection{Factorización LU}

\subsubsection{Descripción del método}

\subsection{Determinación de la Isoterma}

Recordemos que nuestra discretización particiona una sección circular del Alto Horno de la siguiente forma:
 \begin{itemize}
 	\item $0 = \theta_0 < \theta_1 < ... < \theta_n = 2\pi$ en $n$ \'angulos discretos, y
 	\item $r_i = r_0 < r_1 < ... < r_m = r_e$ en $m+1$ radios discretos
 \end{itemize}

Luego, para cada ángulo $j$ tenemos los puntos: $t_{i,j}$ con $0 \leq i \leq m$.

Entonces, hallar la isoterma $C$ equivale a, para cada ángulo $j$, hallar el radio $r_C$ tal que $T(r_C, \theta_j) = C$.

\subsubsection{Promedio simple}

Este método consiste en, dado un ángulo $j$, buscar un punto $t_{i,j}$ en la solución del sistema tal que $t_{i,j} \leq C \leq t_{i+1,j}$.

Una vez hallado este punto, tenemos que $r_C = \frac{r_i + r_{i+1}}{2}$.

\subsubsection{Búsqueda binaria mediante sistemas de ecuaciones}

\subsubsection{Regresión lineal (Linear fit)}

Este método utiliza el algoritmo de regresión lineal para, dado un ángulo $j$, y usando todos los puntos $t_{i,j}$ con $0 \leq i \leq m$, hallar una función lineal que aproxime dichos puntos lo mejor posible.
Como la función que estamos buscando es lineal, es de la forma: $y(x) = a + bx$, donde $b$ es el coeficiente principal, $a$ el termino independiente, $x$ es un radio sobre el ángulo $j$ e $y(x)$ es la temperatura para dicho radio.

Luego, el algoritmo de regresion lineal basicamente utiliza la minimización de la suma de las distancias al cuadradado desde los puntos a la función lineal. Esto se logra calculando la derivada con respecto a $a$ y $b$ y fijando estos en cero.

Entonces, si definimos:

$$\overline{x} = \frac{1}{m}\sum_{i=0}^{m}{r_i} \quad\quad\quad \overline{y} = \frac{1}{m}\sum_{i=0}^{m}{t_{i,j}}$$

$$S_x = \sum_{i=0}^{m}{(r_i - \overline{x})^2} \quad\quad\quad S_{xy} = \sum_{i=0}^{m}{(r_i - \overline{x})(t_{i,j} - \overline{y})}$$

Tenemos que:

$$b = \frac{S_{xy}}{S_x}  \quad\quad\quad a = \frac{\overline{y} - b\overline{x}}{m}$$

Una vez obtenidos $a$ y $b$, para hallar la isoterma $C$ en el ángulo $j$, basta con calcular:

$$r_C = |C - a|/b$$

En pseudocódigo:

\begin{lstlisting}[mathescape=true]
Poner solucion = vector de n elementos
Para j desde 0 hasta n hacer:
    Poner avgX = 0
    Poner avgY = 0
    Para i desde 0 hasta m hacer:
        Poner avgX = avgX + $r_i$
        Poner avgY = avgY + $t_{i,j}$
    Fin para
    Poner avgX = avgX / m
    Poner avgY = avgY / m
    Poner numerador = 0
    Poner denominador = 0
    Para i desde 0 hasta m hacer:
        Poner numerador = numerador + ($r_i$ - avgX) * ($t_{i,j}$ - avgY)
        Poner denominador = denominador + ($r_i$ - avgX) * ($r_i$ - avgX)
    Fin para
    Si denominador == 0 hacer:
        Poner denominador = 1
    Fin si
    Poner coeficiente = numerador / denominador
    Poner independiente = (avgY - slope * coeficiente) / m
    Poner solucion[j] = abs(C - independiente) / coeficiente
Fin para
\end{lstlisting}

\subsection{Evaluación del peligro de la estructura}

Una vez obtenida la isoterma $C$, queremos evaluar la peligrosidad de la estructura en función de la distancia de la isoterma a la pared externa del horno. En este sentido, estamos asumiendo que la temperatura $C$ es elevada y que mientras más cercana está la temperatura de la pared externa a $C$, entonces más peligrosa es la estructura.

En base a esto, proponemos dos medidas distintas para evaluar la peligrosidad.

\subsubsection{Proximidad porcentual simple}

Para cada ángulo $j$, podemos calcular el coeficiente porcentual $\Delta_j(C) = (r_e - r_C)/(r_e - r_i)$, donde $r_e$ es el radio de la pared externa del horno, $r_i$ el radio de la pared interna, y $r_C$ el radio de la isoterma $C$ para el ángulo $j$.

Notese que $r_i \leq r_C \leq r_e$, y por lo tanto si $r_C = r_i \implies \Delta_j(C) = 1$, y si $r_C = r_e \implies \Delta_j(C) = 0$.

De esta forma, podemos definir un $\varepsilon_C$, con $0 < \varepsilon_C < 1 $, tal que decimos que la estructura se encuentra en peligro si:

$$\varepsilon_C \geq \min\limits_{\substack{1 \leq j \leq n-1}}(\Delta_j(C))$$

\subsubsection{Proximidad porcentual promediada}

En la medida anterior, podría pasar que para un $j'$ dado $\Delta_{j'}(C) < \varepsilon_C$ pero el resto de los $\Delta_j(C)$ sean mayores a $\varepsilon_C$, en cuyo caso, igualmente la estructura sería catalogada como peligrosa.

Entonces, querriamos dar una medida de la peligrosidad de la estructura que tome en cuenta todos los angulos. Para esto, vamos a tomar el promedio de todos los $\Delta_j(C)$, definidos como en la medida anterior para cada ángulo $j$, y decimos que la estructura se encuentra en peligro si:

$$\Delta(C)= \frac{\sum\limits_{j=1}^{n}{\Delta_j(C)}}{n} \leq  \varepsilon_C$$
