\begin{proposition}
    Sea $A \in \mathbb{R}^{n(m+1) \times n(m+1)}$ la matriz obtenida para el sistema definido por las ecuaciones del Modelo, en donde $m+1$ y $n$ corresponden a la cantidad de radios y angulos respectivamente de la discretizacion. Demostrar que es posible aplicar Eliminación Gaussiana sin pivoteo.
\end{proposition}

Para poder demostrar la proposición, utilizamos los siguientes lemas, que demostramos a continuacion:

  \begin{enumerate}[label=(\subscript{L}{\arabic*})]
    \item $A$ es una matriz banda.
    \item $A$ es diagonal dominante (no estricta).

  \end{enumerate}

  \begin{customlemma}{$L_{1}$}
    $A$ es una matriz banda
  \end{customlemma}

  \begin{proof}
    A partir del Modelo descripto en el punto anterior, vease figura \ref{fig:matriz}, podemos concluir que $A$ es una matriz banda.
  \end{proof}

  \begin{customlemma}{$L_{2}$}
    $A$ es diagonal dominante (no estricta).
  \end{customlemma}

  \begin{proof}
    Por definición, una matriz es diagonal dominante (no estrictamente) cuando se cumple que, $\forall i = 0,1,...,n-1$:
    \begin{equation*}
        \begin{aligned}
          |a_{i,i}| &\geq \sum\limits_{\substack{j=0  \\ j \neq i}}^{n-1} |a_{i,j}| \\
        \end{aligned}
    \end{equation*}
    Esta desigualdad es evidente para las primeras  y ultimas $n$ filas, ya que el unico valor distinto de 0 se encuentra en la diagonal.
    Falta ver el caso para el resto de $A$. Tenemos que probar que para fila, $i$, se cumple:
    \begin{equation*}
        \begin{aligned}
          |-2\alpha-2\beta+\gamma| \geq |\beta - \gamma| + |\alpha| + |\alpha| + |\beta|
        \end{aligned}
    \end{equation*}
    \newline
    Recordando que: $\alpha = \frac{1}{(\Delta\theta)^2 * r^2}$, $\beta = \frac{1}{(\Delta\ r)^2}$ y $\gamma = \frac{1}{(\Delta\ r) * r}$.
    \newline
    \newline
    Entonces, por definicion sabemos que $|\alpha| = \alpha$ y $|\beta| = \beta$.
    \newline
    \newline
    Veamos que $|\beta - \gamma| = \beta - \gamma$. Supongamos que $\beta - \gamma < 0$:

    \begin{equation*}
        \begin{aligned}
          \beta &< \gamma \\
          %\text{Por definicion de  $\beta$ y $\gamma$:} \\
          \frac{1}{(\Delta\ r)^2}  &< \frac{1}{(\Delta\ r) * r_{j}} \\
          1 &< \frac{\Delta\ r}{r_{j}} \\
          %\text{Sabemos que } \Delta\ r = r_{j} - r_{j-1} \text{ para $j=1,...,m$. Reemplazando:} \\
          1 &< \frac{r_{j} - r_{j-1}}{r_{j}} \\
          1 &< 1 - \frac{r_{j-1}}{r_{j}} \\
          %\text{ Como los radios son todos mayores a 0, llegamos a un absurdo, que vino de suponer }\beta - \gamma &< 0, \text{por lo tanto }\beta - \gamma \geq 0.
        \end{aligned}
    \end{equation*}
    Como los radios son todos mayores a 0, llegamos a un absurdo, que vino de suponer $\beta - \gamma < 0$, por lo tanto $\beta - \gamma \geq 0$.
    \newline
    \newline
    Deberiamos probar que la desigualdad se cumple para los siguientes casos:

    \begin{enumerate}
      \item $-2\alpha-2\beta+\gamma \geq 0$
      \item $-2\alpha-2\beta+\gamma < 0$
    \end{enumerate}
    Veamos caso por caso:
    \begin{enumerate}
      \item \bm{$-2\alpha-2\beta+\gamma \geq 0$}:

        \begin{equation*}
          \begin{aligned}
          -2\alpha-2\beta+\gamma &\geq \beta - \gamma + \alpha + \alpha + \beta \\
           2\gamma &\geq 4\beta + 4\alpha \\
           \gamma &\geq 2\beta + 2\alpha \\
           \gamma - 2\beta - 2\alpha& \geq 0
          \end{aligned}
        \end{equation*}
        Que vale por ser exactamente la hipótesis del caso 1.

      \item \bm{$-2\alpha-2\beta+\gamma < 0$}:

        \begin{equation*}
          \begin{aligned}
           -2\alpha-2\beta+\gamma &\leq -\beta+\gamma-\alpha-\alpha-\beta \\
           \gamma - \gamma &\leq 2\beta - 2\beta + 2\alpha - 2\alpha \\
           0 &\leq 0
          \end{aligned}
        \end{equation*}

        Que vale siempre.

    \end{enumerate}


  \end{proof}

\begin{proof}[Demostración Proposición 1.]

Por $L_{2}$ sabemos que $A$ es diagonal dominante (no estricta) y por definicion del Modelo sabemos que $a_{0,0} = 1$.

Sea $A^{(1)}$ la matriz resultante luego de aplicar un paso de la Eliminación Gaussiana. Para toda fila $i = 1,...,n-1$ se cumple que:

\begin{equation*}
    \begin{aligned}
      a^{(1)}_{i,j} &= a^{(0)}_{i,j} - \frac{a^{(0)}_{0,j}a^{(0)}_{i,j}}{a^{(0)}_{0,0}}, \text{para } 1 \leq j \leq n(m+1)-1
    \end{aligned}
\end{equation*}

Sabemos que $a^{(1)}_{i,0} = 0$. Luego:

\begin{equation*}
    \begin{aligned}
      \sum\limits_{\substack{j=1  \\ j \neq i}}^{n(m+1)-1} |a^{1}_{i,j}| &= \sum\limits_{\substack{j=1  \\ j \neq i}}^{n(m+1)-1} |a^{(0)}_{i,j} - \frac{a^{(0)}_{0,j}a^{(0)}_{i,0}}{a^{(0)}_{0,0}}| \\
      &\leq \sum\limits_{\substack{j=1  \\ j \neq i}}^{n(m+1)-1} |a^{(0)}_{i,j}| + \sum\limits_{\substack{j=1  \\ j \neq i}}^{n(m+1)-1} |\frac{a^{(0)}_{0,j}a^{(0)}_{i,0}}{a^{(0)}_{0,0}}| \\
      &\leq |a^{(0)}_{i,i}| - |a^{(0)}_{i,0}| +  \frac{|a^{(0)}_{i,0}|}{|a^{(0)}_{0,0}|} \sum\limits_{\substack{j=1  \\ j \neq i}}^{n(m+1)-1} |a^{(0)}_{0,j}| \\
      &\leq |a^{(0)}_{i,i}| - |a^{(0)}_{i,0}| +  \frac{|a^{(0)}_{i,0}|}{|a^{(0)}_{0,0}|} (|a^{(0)}_{0,0}| - |a^{(0)}_{0,i}|) \\
      &= |a^{(0)}_{i,i}| - \frac{|a^{(0)}_{i,0}||a^{(0)}_{0,i}|}{|a^{(0)}_{0,0}|} \\
      &\leq |a^{(0)}_{i,i} - \frac{a^{(0)}_{i,0}a^{(0)}_{0,i}}{a^{(0)}_{0,0}}| = |a^{(1)}_{i,i}|
    \end{aligned}
\end{equation*}

\begin{itemize}


\item Por lo tanto, el dominio diagonal no estricto se establece en los renglones $1,..,n-1$, y como el primer renglon de $A^{(1)}$ y de $A$ son iguales,
$A^{(1)}$ sera diagonal dominante no estricto.

\item Para poder aplicar un paso más de la Eliminación Gaussiana, es necesario que $a^{(1)}_{1,1} \neq 0$. Sabemos por definición del Modelo que $a^{(0)}_{1,1} = 1$, y como $a^{(0)}_{1,0} = 0$, por definición de la Eliminación Gaussiana $a^{(1)}_{1,1} = 1$. Esta situación se repite para las $n$ primeras flas de $A$, ya que para $\forall i=0,...,n-1$ $\forall j=0,...,n(m+1)-1 \land j \neq i, a^{(0)}_{i,j} = 0$, por lo tanto podemos afimar que podemos realizar los primeros $n-1$ pasos de la Eliminación Gaussiana, y que la matriz $A^{(n-1)}$ será diagonal dominante no estricta (repitiendo el procedmiento hecho para $A^{(1)}$).

\item Queda ver que sucede para los pasos $n \leq k \leq n(m+1)-n-1$ de la Eliminación Gaussiana. 
  \begin{enumerate}[label=\Roman*]
      \item Para el paso $k=n$, utilizamos que $A$ es una matriz banda ($L_{1}$), en particular la banda esta definida por las filas $i=n,..,n(m+1)-n-1$, en donde el último coeficiente distinto de 0 de cada fila es $\beta_{i}$, el valor de la banda derecha $q$. 
      \item Sabemos que $\beta_{i} > 0$, por definición de los $\beta$. Al realizar el paso $k$, sabemos que la matriz resultante $A^{(k)}$ será diagonal dominante no estricta, (mismo procedimientos que en las filas precedentes) y también sabemos que $\beta^{k}_{k} = \beta^{0}_{k}$, debido al hecho que $\forall u=0,..,k-1, a^{(u)}_{u,j} = 0$, donde $j$ es el indice de la columna de $\beta^{0}_{k}$.
      \item Como $A^{(k)}$ es diagonal dominante no estricta, sabemos que $a^{(k)}_{k,k} \geq \beta^{k}_{k}$, y como $\beta^{k}_{k} > 0$, entonces $a^{(k)}_{k,k} > 0$, por lo tanto podemos realizar un paso más de la Eliminación Gaussiana. 
      \item Esta situación de repite para el resto de las pasos $n+1 \leq k \leq n(m+1)-n-1$.
  \end{enumerate}  
\item Por último queda por ver que sucede con los pasos $ n(m+1)-n \leq k \leq n(m+1)-1$. Esta situación es idéntica al de los primeros $n$ pasos, ya que en $A$  $\forall i=n(m+1)-n,...,n(m+1)-1$ $\forall j=0,...,n(m+1)-1 \land j \neq i, a^{(0)}_{i,j} = 0$. Es decir, el valor de la diagonal de estas filas no será alterado por la Eliminación Gaussiana, y como en $A$ su valor es $1$, podemos aplicar los pasos de la Eliminación Gaussiana.

\item Por todo lo expuesto, podemos concluir que es posible aplicar a $A$ Eliminación Gaussiana sin pivoteo.

\end{itemize}

\end{proof}
