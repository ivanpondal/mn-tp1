\begin{proposition}
    Sea $A \in \mathbb{R}^{n \times n}$ la matriz obtenida para el sistema definido por las ecuaciones del Modelo, en donde $u$ y $v$ corresponden a la cantidad de radios y angulos respectivamente de la discretizacion, tal que $n = uv$. Demostrar que es posible aplicar Eliminación Gaussiana sin pivoteo.
\end{proposition}
%\begin{proof}[Demostración]
Para poder demostrar la proposición, utilizamos los siguientes lemas, que demostramos a continuacion:

  \begin{enumerate}[label=(\subscript{L}{\arabic*})]
    \item $A$ es una matriz banda.
    \item $A$ es diagonal dominante (no estricta).
    %\item En cada paso $k$, con $0 \leq k \leq n-1$, de la Eliminación Gaussiana, para cada fila $i$ de $A^{(k)}$, existe, al menos, un $a_{i,j}$ $\neq$ 0, con $j = 0,..., n-1$
  \end{enumerate}

%\end{proof}

  \begin{lemma}
    $A$ es una matriz banda
  \end{lemma}

  \begin{proof}
    MUESTRO LA MATRIZ Y SU FORMA
  \end{proof}

  \begin{lemma}
    $A$ es diagonal dominante (no estricta).
  \end{lemma}

  \begin{proof}
    Por definición, una matriz es diagonal dominante (no estricamente) cuando se cumple que, $\forall i = 0,1,...,n-1$:
    \begin{equation*}
        \begin{aligned}
          |a_{i,i}| &\geq \sum\limits_{\substack{j=0  \\ j \neq i}}^{n-1} |a_{i,j}| \\
        \end{aligned}
    \end{equation*}
    Esta desigualdad es evidente para las primeras  y ultimas $v$ filas, ya que el unico valor distinto de 0 se encuentra en la diagonal.
    Falta ver el caso para el resto de $A$. Tenemos que probar que para fila, se cumple:
    \begin{equation*}
        \begin{aligned}
          |-2\alpha-2\beta+\gamma| \geq |\beta - \gamma| + |\alpha| + |\alpha| + |\beta|
        \end{aligned}
    \end{equation*}
    Por definicion sabemos que $|\alpha| = \alpha$ y $|\beta| = \beta$
    \newline
    COMPLETAR
  \end{proof}

\begin{proof}[Demostración]

Por $L_{2}$ sabemos que $A$ es diagonal dominante (no estricta) y por definicion del Modelo sabemos que $a_{0,0} = 1$.

Sea $A^{(1)}$ la matriz resultante luego de aplicar un paso de la Eliminación Gaussiana. Para toda fila $i = 1,...,n-1$ se cumple que:

\begin{equation*}
    \begin{aligned}
      a^{(1)}_{i,j} &= a^{(0)}_{i,j} - \frac{a^{(0)}_{0,j}a^{(0)}_{i,j}}{a^{(0)}_{0,0}}, para 1 \leq j \leq n-1
    \end{aligned}
\end{equation*}

Sabemos que $a^{(1)}_{i,0} = 0$. Luego:

\begin{equation*}
    \begin{aligned}
      \sum\limits_{\substack{j=1  \\ j \neq i}}^{n-1} |a^{1}_{i,j}| &= \sum\limits_{\substack{j=1  \\ j \neq i}}^{n-1} |a^{(0)}_{i,j} - \frac{a^{(0)}_{0,j}a^{(0)}_{i,0}}{a^{(0)}_{0,0}}| \\
      &\leq \sum\limits_{\substack{j=1  \\ j \neq i}}^{n-1} |a^{(0)}_{i,j}| + \sum\limits_{\substack{j=1  \\ j \neq i}}^{n-1} |\frac{a^{(0)}_{0,j}a^{(0)}_{i,0}}{a^{(0)}_{0,0}}| \\
      &\leq |a^{(0)}_{i,i}| - |a^{(0)}_{i,0}| +  \frac{|a^{(0)}_{i,0}|}{|a^{(0)}_{0,0}|} \sum\limits_{\substack{j=1  \\ j \neq i}}^{n-1} |a^{(0)}_{0,j}| \\
      &\leq |a^{(0)}_{i,i}| - |a^{(0)}_{i,0}| +  \frac{|a^{(0)}_{i,0}|}{|a^{(0)}_{0,0}|} (|a^{(0)}_{0,0}| - |a^{(0)}_{0,i}|) \\
      &= |a^{(0)}_{i,i}| - \frac{|a^{(0)}_{i,0}||a^{(0)}_{0,i}|}{|a^{(0)}_{0,0}|} \\
      &\leq |a^{(0)}_{i,i} - \frac{a^{(0)}_{i,0}a^{(0)}_{0,i}}{a^{(0)}_{0,0}}| = |a^{(1)}_{i,i}|
    \end{aligned}
\end{equation*}

Por lo tanto, el dominio diagonal no estricto se establece en los renglones $1,..,n-1$, y como el primer renglon de $A^{(1)}$ y de $A$ son iguales,
$A^{(1)}$ sera diagonal dominante no estricto.

Este proceso lo podemos repetir hasta obtener $A^{(n-1)}$ que sera diagonal dominante no estricto.

Sin embargo, falta ver que para cada paso, $k$, de la Eliminación Gaussiana, el elemento de la diagonal, $a_{k,k}$ de $A^{(k)}$, es distinto de 0.
Para poder demostrar esto, usamos $L_{2}$ y la forma especifica de $A$ dada por el Modelo. Veamos los distintos casos:

\begin{itemize}
  \item Para $0 \leq k \leq v-1$, sabemos que $a^{(k)}_{k,k} = a^{(0)}_{k,k} = 1$, ya que $\forall j=0,...,n-1 \land j \neq k, a^{(k)}_{k,j} = 0$,
  es decir los valores de la diagonal de las $v$ primeras filas no cambian en los primeros $v$ pasos de la Eliminación Gaussiana, debido a que
  esas filas solo tienen un valor distinto de 0 en la diagonal, valor que no es afectado por las filas que las preceden, por definicion del procedimiento de Eliminación Gaussiana.
  \item Para $v \leq k \leq n-v-1$, sabemos que para $k = v$, $a^{(k)}_{k,k} = a^{(0)}_{k,k}$, (mismo argumento del punto anterior), basta ver que sucede en el resto de los casos.
  Sabemos, por la forma del Modelo, que esta franja de filas definen la banda de $A^{(0)}$, en particular la banda $q$ en $A^{(0)}$ esta compuesta por los valores $\beta_{i}$, donde $i$ representa su fila.
  Por lo tanto estos valores no varian cuando $k \leq i$, es decir $a^{(k)}_{i,j} = a^{(0)}_{i,j}$ donde $j$ es la columna de $\beta_{i}$ , y como ya vimos que todos los $\beta_{i} \neq 0$, sumado al hecho, ya demostrado,
  que se preserva la caracteristica de diagonal dominante, podemos concluir que $a^{(k)}_{k,k} \neq 0$.
  \item Por ultimo para $n-v \leq k \leq n-1$, tenemos la misma situacion que las primeras $v$ filas.
\end{itemize}

Habiendo demostrado que los elementos de la diagonal, en cada paso de la Eliminación Gaussiana, son distintos de 0, podemos concluir que es posible aplicar Eliminación Gaussiana sin pivoteo.

\end{proof}

%  \begin{lemma}
%    En cada paso $k$, con $0 \leq k \leq n-1$, de la Eliminación Gaussiana, para cada fila $i$ de $A^{(k)}$, existe, al menos, un $a_{i,j}$ $\neq$ 0, con $j = 0,..., n-1$
%  \end{lemma}

%  \begin{proof}

%    Lo probamos por Induccion en la cantidad $k$ de pasos de la Eliminación Gaussiana, utilizando $L_{1}$.
%    \newline
%    Caso base: $A^{(0)} = A$
%    \begin{enumerate}
%      \item Para las primeras y ultimas $u$ filas, el elemento de la diagonal es 1.
%      \item Para el resto de las filas, basta ver que algun valor es distinto de 0. Por definicion sabemos que $\beta \neq 0$
%    \end{enumerate}
%    Paso Inductivo: Queremos ver que en el paso $k$ de la Eliminación Gaussiana, para cada fila $i$ de $A^{(k)}$, existe, al menos, un $a_{i,j}$ $\neq$ 0, con $j = 0,..., n-1$
%    \newline
%    Sabemos, por definición del procedimiento de Eliminación Gaussiana, que las filas $i = 0,..,k$ de $A^{(k)}$ son iguales a las de $A^{(k-1)}$, por lo tanto, por Hipotesis Inductiva,
%    existe, al menos, un $a_{i,j}$ $\neq$ 0, con $j = 0,..., n-1$.
%    \newline
%    Queda por ver que sucede con las filas $i = k+1,..n-1$ de $A^{(k)}$. Tenemos dos casos:
%    \begin{enumerate}
%      \item Filas que tienen un valor ($\beta$) como extremo derecho de la banda
%      \item Filas que no tienen valores en los extremos de las bandas (las filas correspondientes a las $u$ primeras y ultimas de $A$)
%    \end{enumerate}
%    COMPLETAR (POSIBLE PROBLEMA)
 %\end{proof}
