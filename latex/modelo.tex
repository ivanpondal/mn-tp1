\subsection{Descripción}

El Alto Horno está definido por las siguientes variables:
\begin{itemize}
    \item El radio de la pared exterior: $r_e \in \mathbb{R}$
    \item El radio de la pared interior: $r_i \in \mathbb{R}$
    \item La temperatura en cada punto de la pared:  $T(r,\theta)$, donde $(r,\theta)$ se encuentra expresado en coordenadas polares, siendo $r$ el radio y $\theta$ el \'angulo polar de dicho punto.

    Son datos del problema, las temperaturas de la pared interior y exterior:
    \begin{itemize}
        \item $T(r_i,\theta) = T_i \;\;\;\;\;para\;todo\;punto\;(r,\theta)\;con\;r\leq r_i$
        \item $T(r_e,\theta) = T_e(\theta) \;\;\;\;\;\;para\;todo\;punto\;(r_e,\theta)$
    \end{itemize}
\end{itemize}

La Figura 1 muestra las variables al tomar una sección circular del horno.

\begin{figure}[ht]
\begin{center}
\includegraphics[width=0.6\columnwidth]{imagenes/horno.png}
\caption{Secci\'on circular del horno}
\end{center}
\end{figure}

En el estado estacionario, cada punto de la pared satisface la ecuación del calor:

\begin{equation}\label{calor}
\frac{\partial^2T(r,\theta)}{\partial r^2}+\frac{1}{r}\frac{\partial T(r,\theta)}{\partial r}+\frac{1}{r^2}\frac{\partial^2T(r,\theta)}{\partial \theta^2} = 0
\end{equation}

\medskip

Para resolver este problema computacionalmente, discretizamos el dominio del problema (el sector A) en coordenadas polares. Consideramos una partici\'on $0 = \theta_0 < \theta_1 < ... < \theta_n = 2\pi$ en $n$ \'angulos discretos con $\theta_k-\theta_{k-1} = \Delta\theta$ para $k = 1,...,n$, y una partici\'on $r_i = r_0 < r_1 < ... < r_m = r_e$ en $m+1$ radios discretos con $r_j - r_{j-1} = \Delta r$ para $j = 1,...,m$.

\medskip

El problema ahora consiste en determinar el valor de la funci\'on $T$ en los puntos de la discretizaci\'on $(r_j,\theta_k)$ que se encuentren dentro del sector A. Llamemos $t_{j,k} = T(r_j,\theta_k)$ al valor (desconocido) de la funci\'on $T$ en el punto $(r_j,\theta_k)$.

\medskip

Para encontrar estos valores, transformamos la ecuaci\'on (\ref{calor}) en un conjunto de ecuaciones lineales sobre las inc\'ognitas $t_{j,k}$, evaluando (\ref{calor}) en todos los puntos de la discretizaci\'on que se encuentren dentro del sector A. Al hacer esta evaluaci\'on, aproximamos las derivadas parciales de $T$ en (\ref{calor}) por medio de las siguientes f\'ormulas de diferencias finitas:


\begin{equation}
\frac{\partial^2T(r,\theta)}{\partial r^2}(r_j,\theta_k) \cong \frac{t_{j-1,k}-2t_{j,k}+t_{j+1,k}}{(\Delta r)^2}
\end{equation}

\begin{equation}
\frac{\partial T(r,\theta)}{\partial r}(r_j,\theta_k) \cong \frac{t_{j,k}-t_{j-1,k}}{\Delta r}
\end{equation}

\begin{equation}
\frac{\partial^2T(r,\theta)}{\partial \theta^2}(r_j,\theta_k) \cong \frac{t_{j,k-1}-2t_{j,k}+t_{j,k+1}}{(\Delta \theta)^2}
\end{equation}

Luego,

\begin{equation}\label{calor_discretizada}
\frac{\partial^2T(r,\theta)}{\partial r^2}+\frac{1}{r}\frac{\partial T(r,\theta)}{\partial r}+\frac{1}{r^2}\frac{\partial^2T(r,\theta)}{\partial \theta^2} \cong \frac{t_{j-1,k}-2t_{j,k}+t_{j+1,k}}{(\Delta r)^2} + \frac{1}{r}\frac{t_{j,k}-t_{j-1,k}}{\Delta r} + \frac{1}{r^2}\frac{t_{j,k-1}-2t_{j,k}+t_{j,k+1}}{(\Delta \theta)^2}
\end{equation}

Si agrupamos los términos para los $t$, la ecuación (\ref{calor_discretizada}) nos queda:

\begin{equation}
   \begin{aligned}
    &(\frac{1}{(\Delta\ r)^2} - \frac{1}{(\Delta\ r) * r})t_{j-1,k} + (\frac{1}{(\Delta\theta)^2 * r^2})t_{j,k-1} +  (-\frac{2}{(\Delta\theta)^2 * r^2}-\frac{2}{(\Delta\ r)^2} +\\
    &(\frac{1} {(\Delta\ r) * r})t_{j,k} + (\frac{1}{(\Delta\theta)^2 * r^2})t_{j,k+1} + (\frac{1}{(\Delta\ r)^2})t_{j+1,k} = 0
    \end{aligned}
\end{equation}

\subsection{Representación del sistema}

Tenemos:
\begin{itemize}
    \item $0 < j < m+1$ : los índices de los radios
    \item $0 < k < n$ : los índices de los ángulos
    \item $\alpha_{j,k} = \frac{1}{(\Delta\theta)^2 * r_{j}^2}$ donde $\Delta\theta = \theta_{k}-\theta_{k-1}$
    \item $\beta_{j,k} = \frac{1}{(\Delta\ r)^2}$ donde $\Delta r = r_{j}-r_{j-1}$
    \item $\gamma_{j,k} = \frac{1}{(\Delta\ r) * r_j}$ donde $\Delta r = r_{j}-r_{j-1}$
\end{itemize}

Entonces, a partir de la ecuación previa (6), podemos reemplazar usando los términos nuevos:
\begin{equation}
   \begin{aligned}
      (\beta_{j,k} - \gamma_{j,k})t_{j-1,k} +
      (\alpha_{j,k})t_{j,k-1} +
      (-2\alpha_{j,k}-2\beta_{j,k}+\gamma_{j,k})t_{j,k} +
      (\alpha_{j,k})t_{j,k+1} +
      (\beta_{j,k})t_{j+1,k} &= 0\\
    \end{aligned}
\end{equation}

A partir de lo detallado previamente, armamos el siguiente sistema de ecuaciones:
\begingroup
    \fontsize{8pt}{10pt}\selectfont
    \begin{equation*}
        \begin{aligned}
          t_{0,0} &= T_{i}(0) \\
          ...& \\
          t_{0,n-1} &= T_{i}(n-1) \\
          (\beta_{1,0} - \gamma_{1,0})t_{0,0} +
          (\alpha_{1,0})t_{1,n-1} +
          (-2\alpha_{1,0}-2\beta_{1,0}+\gamma_{1,0})t_{1,0} +
          (\alpha_{1,0})t_{1,1} +
          (\beta_{1,0})t_{2,0} &= 0\\
          ...& \\
          (\beta_{j,k} - \gamma_{j,k})t_{j-1,k} +
          (\alpha_{j,k})t_{j,k-1} +
          (-2\alpha_{j,k}-2\beta_{j,k}+\gamma_{j,k})t_{j,k} +
          (\alpha_{j,k})t_{j,(k+1)\%n} +
          (\beta_{j,k})t_{j+1,k} &= 0\\
          ...& \\
          (\beta_{m-1,n-1} - \gamma_{m-1,n-1})t_{m-2,n-1} +
          (\alpha_{m-1,n-1})t_{m-1,n-2} + \\
          (-2\alpha_{m-1,n-1}-2\beta_{m-1,n-1}+\gamma_{m-1,n-1})t_{m-1,n-1} +
          (\alpha_{m-1,n-1})t_{m-1,0} +
          (\beta_{m-1,n-1})t_{m,n-1} &= 0\\
          t_{m,0} &= T_{e}(0) \\
          ...& \\
          t_{m,n-1} &= T_{e}(n-1) \\
        \end{aligned}
    \end{equation*}
  \endgroup
En donde las primeras $n$ ecuaciones, son las ecuaciones correspondientes al radio de la pared interior, $r_{i}$, para los distintos $\theta$ de la discretización. Luego para cada radio $r \neq r_{i}$ y $r \neq r_{e}$ , se listan las ecuaciones correspondientes a los distintos $\theta$ de la discretización. Por último las últimas $n$ ecuaciones son las ecuaciones correspondientes al radio de la pared exterior, $r_{e}$, para los distintos $\theta$ de la discretización.
\newline
\newline
Para representar el sistema de ecuaciones presentado, se utilizará una matriz cuadrada simple de tamaño $n(m+1)$, en donde las filas representan las ecuaciones detalladas previamente y las columnas cada punto de la discretización, $t_{j,k}$, implementada como un vector de vectores. A continuación, se muestra como quedaría la matriz para las ecuaciones de los puntos $t_{0,0}, t_{0,n-1}, t_{j,k}, t_{m,0}$ y $t_{m,n-1}$.

\[
    % reduce columns padding:
    \setlength{\arraycolsep}{1pt}
    % fila vertical alternativa:
    % \vdots   &  \multicolumn{6}{c}{\strut}      &\vdots&         \multicolumn{6}{c}{}               & \vrule & {} \\
    \kbordermatrix{%
                       & \bm{t_{0,0}} & \ldots & \bm{t_{0,n-1}} & \ldots & t_{j-1,k}           & \ldots & t_{j,k-1}   & \bm{t_{j,k}}                & t_{j,k+1}   & \ldots & t_{j+1,k}   & \ldots & \bm{t_{m,0}} & \ldots & \bm{t_{m,n-1}} &        & b        \\
        \bm{t_{0,0}}   & 1            & \ldots & 0              & \ldots & 0                   & \ldots & 0           & 0                           & 0           & \ldots & 0           & \ldots & 0            & \ldots & 0              & \vrule & t_{0,0}   \\
        \vdots         & \vdots       & {}     & \vdots         & {}     & \vdots              & {}     & \vdots      & \vdots                      & \vdots      & {}     & \vdots      & {}     & \vdots       & {}     & \vdots         & \vrule & {}       \\
        \bm{t_{0,n-1}} & 0            & \ldots & 1              & \ldots & 0                   & \ldots & 0           & 0                           & 0           & \ldots & 0           & \ldots & 0            & \ldots & 0              & \vrule & t_{0,n-1} \\
        \vdots         & \vdots       & {}     & \vdots         & {}     & \vdots              & {}     & \vdots      & \vdots                      & \vdots      & {}     & \vdots      & {}     & \vdots       & {}     & \vdots         & \vrule & {}       \\
        \bm{t_{j,k}}   & 0            & \ldots & 0              & \ldots & \bm{\beta_{j,k} - \gamma_{j,k}} & \ldots & \bm{\alpha_{j,k}} & \bm{-2\alpha_{j,k}-2\beta_{j,k}+\gamma_{j,k}} & \bm{\alpha_{j,k}} & \ldots & \bm{\beta_{j,k}}  & \ldots & 0            & \ldots & 0              & \vrule & 0        \\
        \vdots         & \vdots       & {}     & \vdots         & {}     & \vdots              & {}     & \vdots      & \vdots                      & \vdots      & {}     & \vdots      & {}     & \vdots       & {}     & \vdots         & \vrule & {}       \\
        \bm{t_{m,0}}   & 0            & \ldots & 0              & \ldots & 0                   & \ldots & 0           & 0                           & 0           & \ldots & 0           & \ldots & 1            & \ldots & 0              & \vrule & t_{m,0}   \\
        \vdots         & \vdots       & {}     & \vdots         & {}     & \vdots              & {}     & \vdots      & \vdots                      & \vdots      & {}     & \vdots      & {}     & \vdots       & {}     & \vdots         & \vrule & {}       \\
        \bm{t_{m,n-1}} & 0            & \ldots & 0              & \ldots & 0                   & \ldots & 0           & 0                           & 0           & \ldots & 0           & \ldots & 0            & \ldots & 1              & \vrule & t_{m,n-1}
    }
\]
\captionof{figure}{Matriz del Sistema}\label{fig:matriz}

\medskip
Notese que, para las filas que representan las ecuaciones de los puntos de la pared distintos del interior e exterior, los coeficientes distintos de 0 se ubican en $t_{j-1,k}$, $t_{j,k}$, $t_{j+1,k}$, $t_{j,k-1}$ y $t_{j,k+1}$.
Es importante destacar que el primer coeficiente distinto de 0 es siempre $\beta_{j,k} - \gamma_{j,k}$, mientras que el ultimo es siempre $\beta_{j,k}$, siendo la distancia entre ellos $2n$. Esto se debe al hecho de que, en nuestra disposicion de las ecuaciones del sistema en la matriz,
en las columnas fijamos el radio y luego avanzamos con los distintos angulos para ese radio, para luego avanzar de radio, por lo cual $t_{j-1,k}$ y $t_{j+1,k}$ seran siempre el primer e ultimo coeficiente distinto de 0 en ese tipo de fila.

Esta caracteristica de la matriz la hace una matriz banda, en donde las diagonales $p, q = n$, estan compuestas por los valores $\beta_{j,k} - \gamma_{j,k}$ y $\beta_{j,k}$ respectivamente.
\newline
\newline
\textbf{Ejemplo con} $\bm{n,m+1=3}$:
\begingroup
    \fontsize{7pt}{7pt}\selectfont
\[
    % reduce columns padding:
    \setlength{\arraycolsep}{3pt}
    % fila vertical alternativa:
    % \vdots   &  \multicolumn{6}{c}{\strut}      &\vdots&         \multicolumn{6}{c}{}               & \vrule & {} \\
    \kbordermatrix{%
                    & {t_{0,0}}               & {t_{0,1}}              & t_{0,2}                  & t_{1,0}                             & t_{1,1}                             & t_{1,2}                             & t_{2,0}       & t_{2,1}            & t_{2,2}       &        & b        \\
        {t_{0,0}}   & 1                       & 0                       & 0                       & 0                                   & 0                                   & 0                                   & 0             & 0                  & 0             & \vrule & t_{0,0}   \\
        {t_{0,1}}   & 0                       & 1                       & 0                       & 0                                   & 0                                   & 0                                   & 0             & 0                  & 0             & \vrule & t_{0,1}       \\
        {t_{0,2}}   & 0                       & 0                       & 1                       & 0                                   & 0                                   & 0                                   & 0             & 0                  & 0             & \vrule & t_{0,2} \\
        {t_{1,0}}   & \beta_{1,0} - \gamma_{1,0}  & 0                       & 0                       & -2\alpha_{1,0}-2\beta_{1,0}+\gamma_{1,0}  & \alpha_{1,0}                          & \alpha_{1,0}                          & \beta_{1,0}     & 0                  & 0             & \vrule & 0       \\
        {t_{1,1}}   & 0                       & \beta_{1,1} - \gamma_{1,1}  & 0                       & \alpha_{1,1}                          & -2\alpha_{1,1}-2\beta_{1,1}+\gamma_{1,1}  & \alpha_{1,1}                          & 0             & \beta_{1,1}          & 0             & \vrule & 0        \\
        {t_{1,2}}   & 0                       & 0                       & \beta_{1,2} - \gamma_{1,2}  & \alpha_{1,2}                          & \alpha_{1,2}                          & -2\alpha_{1,2}-2\beta_{1,2}+\gamma_{1,2}  & 0             & 0                  & \beta_{1,2}     & \vrule & 0       \\
        {t_{2,0}}   & 0                       & 0                       & 0                       & 0                                   & 0                                   & 0                                   & 1             & 0                  & 0             & \vrule & t_{2,0}   \\
        {t_{2,1}}   & 0                       & 0                       & 0                       & 0                                   & 0                                   & 0                                   & 0             & 1                  & 0             & \vrule & t_{2,1}       \\
        {t_{2,2}}   & 0                       & 0                       & 0                       & 0                                   & 0                                   & 0                                   & 0             & 0                  & 1             & \vrule & t_{2,2}
    }
\]
\endgroup
