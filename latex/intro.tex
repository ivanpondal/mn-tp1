El objetivo de este Trabajo Práctico es implmentar diferentes algoritmos de resolución de sistemas de ecuaciones lineales y experimentar con dichas implementaciones en el contexto de un problema de la vida real.

El problema a resolver es hallar la isoterma 500C en la pared de un Alto Horno. Para tal fin, deberemos particionar la pared del horno en puntos finitos, y luego resolver un sistema de ecuaciones lineales, en el cual cada punto de la pared interior y exterior del Horno es un dato, y las ecuaciones para los puntos internos satisfacen la ecuación del calor.

Los experimentos realizados se dividen en dos partes: Comportamiento del sistema y Evaluación de los métodos. En la primera parte, analizaremos con los  distintas instancias de prueba y se estudiará la proximidad de la isoterma buscada respecto de la pared exterior del horno. En la segunda parte, analizaremos el tiempo de computo requerido para la resolución del sistema en función de la granularidad de la discretización y analizaremos el escenario en el cual las temperaturas de los bordes varian a lo largo del tiempo.
