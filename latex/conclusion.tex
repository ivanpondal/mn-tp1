\setlength{\parindent}{15.0pt} % algún comando dejó en cero el parindent
En este trabajo pudimos no solo modelar el sistema planteado, sino que
apreciar y aprovechar las propiedades del mismo para así resolverlo con los
métodos estudiados observando también las características de ellos.

Por un lado mediante la forma en la que construímos nuestro sistema probamos
que se podía resolver con Eliminación Gaussiana sin pivoteo. Además produjimos
una versión mejorada del algoritmo de eliminación donde aprovechando la
propiedad de banda de la matriz del sistema, redujimos drásticamente la
cantidad de operaciones necesarias para resolverla.

Así mismo, cabe destacar que al realizar operaciones con aritmética finita,
tanto para la solución de los sistemas como para el cálculo de la isoterma donde
 la reutilización de datos arrastra error, nosotros no podemos
garantizar que los resultados obtenidos sean exactos, pero dado que
realizamos varias instancias de prueba con distintas
metodologías y tomando números de condición aceptables, pudimos ver que los
valores que obtuvimos eran coherentes a su contexto.

Luego, en lo que respecta el cálculo de la isoterma, al plantear diversas
metodologías tuvimos la posiblidad de analizar y discutir los resultados de las
mismas, donde en particular pudimos observar cómo al utilizar la búsqueda
binaria podíamos llegar al grado de precisión que deseásemos y que para el
método por promedio, al aumentar la cantidad de particiones mejoraba la
aproximación, mientras que usando la regresión lineal, esta se ajustaba más a
una función lineal que no reflejaba el comportamiento de la fórmula de calor,
convergiendo así a un valor distinto tanto al del promedio como el de la
búsqueda binaria.

Mediante estas aproximaciones, habiendo establecido previamente nuestro
criterio para evaluar si una estructura se encontraba en peligro, llegamos a
estimar qué sistemas eran seguros dentro de lo estipulado.

Para el análisis del tiempo de ejecución de una así como
varias instancias del sistema modelado, vimos cómo se cumplían las complejidades
teóricas de la resolución a través de Eliminación Gaussiana y LU. En este
análisis corroboramos cómo si se trataba de una sola instancia la Eliminación
Gaussiana presentaba una ventaja sobre LU, dado que el último debe calcular su
factorización en su primer corrida, mientras que al subir el número de
instancias el algoritmo para LU lograba un tiempo sumamente mejor que el de
Eliminación Gaussiana, ya que con la factorización LU habiendo pagado un costo
cúbico en la primer instancia, luego es es del orden cuadrático contra el
siempre cúbico de la Eliminación Gaussiana. Además en el análisis para el
algoritmo de Eliminación Gaussiana con la optimización de banda llegamos a
concluir que su tiempo de ejecución llegaba a reducirse al de orden cuadrático.

Por último, podemos mencionar algunos experimentos que podrían realizarse a
futuro, como el aprovechamiento de la matriz banda en lo que es el algoritmo
para la factorización LU, ya que esta optimización se realizó sólo para la
Eliminación Gaussiana, junto a su correspondiente estudio de tiempo de
ejecución. A su vez, quedó pendiente el realizar la mejora no únicamente en lo
que son los tiempos de ejecución sino que el espacio que consume nuestro
algoritmo dado que en la matriz banda gran parte de la misma permanece
inalterada. También se podría haber profundizado en la experimentación del
cálculo de la isoterma con sistemas donde la temperatura interna y externa no
fueran constantes si no que tuvieran algún tipo de fluctuación donde se pudiera
ver con más detalle cómo se comportaba cada método.
